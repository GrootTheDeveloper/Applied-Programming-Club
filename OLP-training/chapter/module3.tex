\chapter{CẤU TRÚC RẼ NHÁNH}

\minitoc

\section{Cấu trúc rẽ nhánh \texttt{if -- else} trong C++}

Trong lập trình, rất nhiều bài toán yêu cầu chương trình phải \textbf{ra quyết định}:
nếu điều kiện đúng thì làm A, nếu sai thì làm B.
Trong C++, cấu trúc \texttt{if -- else} là công cụ cơ bản để xử lý những tình huống này.

\subsection{Biểu thức điều kiện là gì?}

\begin{itemize}
    \item \textbf{Biểu thức điều kiện} là một biểu thức có giá trị \texttt{true} (đúng) hoặc \texttt{false} (sai).
    \item Thường là các phép so sánh: \texttt{==}, \texttt{!=}, \texttt{<}, \texttt{>}, \texttt{<=}, \texttt{>=}.
    \item Có thể kết hợp nhiều điều kiện bằng toán tử logic: \texttt{\&\&} (AND), \texttt{||} (OR), \texttt{!} (NOT).
\end{itemize}

Trong C++:
\begin{itemize}
    \item \texttt{0} được xem là \textbf{false}.
    \item Mọi giá trị \textbf{khác 0} được xem là \textbf{true}.
\end{itemize}

Ví dụ, các điều kiện sau đều có kiểu \texttt{bool}:
\begin{lstlisting}
// Check if x equals 10
(x == 10)

// Check if n is even
(n % 2 == 0)

// Check if 20 <= n <= 50
(n >= 20 && n <= 50)

// Check if n is divisible by 3 or 5
(n % 3 == 0 || n % 5 == 0)
\end{lstlisting}

\vspace{2mm}

%------------------------------------------------
\subsection{Câu lệnh \texttt{if}}

Câu lệnh \texttt{if} cho phép bạn \textbf{chỉ thực hiện một khối lệnh nếu điều kiện đúng}.

\subsubsection{Cú pháp cơ bản}

\begin{lstlisting}
if (condition) {
    // code block executed when condition is true
}
\end{lstlisting}

\textbf{Ý nghĩa:}
\begin{itemize}
    \item Nếu \texttt{condition} đúng (\texttt{true}) → khối lệnh bên trong \texttt{\{\}} được chạy.
    \item Nếu \texttt{condition} sai (\texttt{false}) → khối lệnh bị bỏ qua.
\end{itemize}

\vspace{2mm}

\subsubsection{Ví dụ 1: Kiểm tra giá trị cụ thể}

\textbf{Yêu cầu:} Nếu \texttt{n == 28} thì in ra \texttt{APC}.

\begin{lstlisting}
#include <iostream>
using namespace std;

int main() {
    int n = 28;

    // If n equals 28, print "APC"
    if (n == 28) {
        cout << "APC\n";
    }

    // This condition is false, so this line will not be printed
    if (n > 30) {
        cout << "APC Blog\n";
    }

    cout << "end\n";
    return 0;
}
\end{lstlisting}

\textbf{Output:}
\begin{verbatim}
APC
end
\end{verbatim}

\vspace{2mm}

\subsubsection{Ví dụ 2: Kiểm tra số chẵn}

Một số nguyên \textbf{là số chẵn} nếu chia cho 2 dư 0.

\begin{lstlisting}
#include <iostream>
using namespace std;

int main() {
    int n = 28;

    // If n is even, print "CHAN"
    if (n % 2 == 0) {
        cout << "CHAN\n";
    }

    // If n is odd, print "LE"
    if (n % 2 != 0) {
        cout << "LE\n";
    }

    return 0;
}
\end{lstlisting}

\textbf{Output:}
\begin{verbatim}
CHAN
\end{verbatim}

\vspace{2mm}

\subsubsection{Ví dụ 3: Kiểm tra tính chia hết}

\begin{lstlisting}
#include <iostream>
using namespace std;

int main() {
    int n = 28, m = 4;

    // Check if n is divisible by m
    if (n % m == 0) {
        cout << "Chia het\n";
    }

    if (n % m != 0) {
        cout << "Khong chia het\n";
    }

    return 0;
}
\end{lstlisting}

\textbf{Output:}
\begin{verbatim}
Chia het
\end{verbatim}

\vspace{2mm}

\subsubsection{Ví dụ 4: Kết hợp nhiều điều kiện}

\textbf{Yêu cầu:} Nếu \texttt{n} thuộc một trong các số \{2, 3, 5, 7\} thì in \texttt{YES}.

\begin{lstlisting}
#include <iostream>
using namespace std;

int main() {
    int n = 5;

    // Check if n is one of 2, 3, 5 or 7
    if (n == 2 || n == 3 || n == 5 || n == 7) {
        cout << "YES\n";
    }

    return 0;
}
\end{lstlisting}

\textbf{Output:}
\begin{verbatim}
YES
\end{verbatim}

\vspace{2mm}

\subsubsection{Lưu ý: dùng số làm điều kiện}

Vì mọi số khác 0 được xem là \texttt{true}, ta có thể viết:

\begin{lstlisting}
#include <iostream>
using namespace std;

int main() {
    int n = 28;
    int m = 0;

    // n is non-zero -> condition is true
    if (n) {
        cout << "n is non-zero\n";
    }

    // m is zero -> condition is false
    if (m) {
        cout << "m is non-zero\n";
    }

    cout << "END\n";
    return 0;
}
\end{lstlisting}

\textbf{Output:}
\begin{verbatim}
n is non-zero
END
\end{verbatim}

\vspace{3mm}

%------------------------------------------------
\subsection{Câu lệnh \texttt{if -- else}}

Câu lệnh \texttt{if -- else} cho phép ta xử lý \textbf{cả hai nhánh}:
\begin{itemize}
    \item Khi điều kiện đúng
    \item Khi điều kiện sai
\end{itemize}

\subsubsection{Cú pháp}

\begin{lstlisting}
if (condition) {
    // executed when condition is true
} else {
    // executed when condition is false
}
\end{lstlisting}

\textbf{Lưu ý:}
\begin{itemize}
    \item \texttt{else} luôn đi kèm với một \texttt{if} trước đó.
    \item Chỉ \textbf{một} trong hai khối \texttt{if} hoặc \texttt{else} được thực thi.
\end{itemize}

\vspace{2mm}

\subsubsection{Ví dụ 1: Kiểm tra chẵn / lẻ (có \texttt{else})}

\begin{lstlisting}
#include <iostream>
using namespace std;

int main() {
    int n;
    cin >> n;

    // If n is even, print "CHAN" and "APC"
    if (n % 2 == 0) {
        cout << "CHAN\n";
        cout << "APC\n";
    } else {
        // Otherwise, print "LE" and the APC website
        cout << "LE\n";
        cout << "sot.umtoj.edu.vn\n";
    }

    return 0;
}
\end{lstlisting}

\textbf{Ví dụ input:} \texttt{28}

\textbf{Output:}
\begin{verbatim}
CHAN
APC
\end{verbatim}

\vspace{2mm}

\subsubsection{Ví dụ 2: Kiểm tra năm nhuận}

\textbf{Định nghĩa năm nhuận:}
\begin{itemize}
    \item Chia hết cho 400, \textbf{hoặc}
    \item Chia hết cho 4 nhưng \textbf{không} chia hết cho 100.
\end{itemize}

\begin{lstlisting}
#include <iostream>
using namespace std;

int main() {
    int year;
    cin >> year;

    bool isLeap =
        (year % 400 == 0) ||
        (year % 4 == 0 && year % 100 != 0);

    if (isLeap) {
        cout << "YES\n";
    } else {
        cout << "NO\n";
    }

    return 0;
}
\end{lstlisting}

\textbf{Input:}
\begin{verbatim}
2020
\end{verbatim}

\textbf{Output:}
\begin{verbatim}
YES
\end{verbatim}

\vspace{3mm}

%------------------------------------------------
\subsection{Chuỗi \texttt{if -- else if -- else}}

Khi có nhiều trường hợp cần phân loại, ta có thể sử dụng
\textbf{nhiều điều kiện nối tiếp nhau}:

\begin{lstlisting}
if (condition1) {
    // case 1
} else if (condition2) {
    // case 2
} else if (condition3) {
    // case 3
} else {
    // default case
}
\end{lstlisting}

\textbf{Cách hoạt động:}
\begin{itemize}
    \item Các điều kiện được kiểm tra \textbf{từ trên xuống dưới}.
    \item Ngay khi gặp điều kiện đúng, khối lệnh tương ứng được thực thi,
    \item Các điều kiện phía sau \textbf{không được kiểm tra nữa}.
\end{itemize}

\subsubsection{Ví dụ: Phân loại điểm}

\begin{lstlisting}
#include <iostream>
using namespace std;

int main() {
    int score;
    cin >> score;

    if (score >= 90) {
        cout << "Excellent\n";
    } else if (score >= 75) {
        cout << "Good\n";
    } else if (score >= 50) {
        cout << "Average\n";
    } else {
        cout << "Need improvement\n";
    }

    return 0;
}
\end{lstlisting}

\vspace{3mm}

%------------------------------------------------
\subsection{\texttt{if -- else} lồng nhau}

Khi điều kiện phức tạp, ta có thể \textbf{chia nhỏ} bằng cách đặt \texttt{if -- else}
bên trong một \texttt{if -- else} khác.

\subsubsection{Ví dụ: Kiểm tra đoạn và chia hết}

\textbf{Yêu cầu:}
\begin{itemize}
    \item $20 \le n \le 50$
    \item và $n$ chia hết cho ít nhất một trong các số $2,3,5,7$
\end{itemize}

\begin{lstlisting}
#include <iostream>
using namespace std;

int main() {
    int n;
    cin >> n;

    // First, check if n is in the range [20, 50]
    if (n >= 20 && n <= 50) {

        // Then, check divisibility by 2, 3, 5 or 7
        if (n % 2 == 0 || n % 3 == 0 ||
            n % 5 == 0 || n % 7 == 0) {
            cout << "YES\n";
        } else {
            cout << "NO\n";
        }

    } else {
        cout << "NO\n";
    }

    return 0;
}
\end{lstlisting}

\vspace{3mm}

%------------------------------------------------
\subsection{Một số lưu ý quan trọng}

\begin{itemize}
    \item Luôn dùng dấu ngoặc nhọn \texttt{\{\}} cho khối lệnh, kể cả khi chỉ có một dòng,
    để tránh bug khi thêm dòng mới sau này.
    \item Cẩn thận với toán tử gán \texttt{=} và so sánh \texttt{==}.
    \item Nên tách điều kiện phức tạp thành nhiều biến trung gian \texttt{bool}
    để code dễ đọc hơn.
    \item Khi nhiều nhánh \textbf{loại trừ lẫn nhau}, ưu tiên dùng \texttt{if -- else if -- else}
    thay vì nhiều \texttt{if} rời rạc.
\end{itemize}

\textbf{Ví dụ: sử dụng biến \texttt{bool} giúp code dễ đọc hơn}

\begin{lstlisting}
bool inRange = (n >= 20 && n <= 50);
bool divisible = (n % 2 == 0 || n % 3 == 0 ||
                  n % 5 == 0 || n % 7 == 0);

if (inRange && divisible) {
    cout << "YES\n";
} else {
    cout << "NO\n";
}
\end{lstlisting}


\section{\texttt{if} và \texttt{else if} (nhiều nhánh điều kiện)}

Trong nhiều bài toán, chương trình không chỉ có hai trường hợp ``đúng/sai'' mà có thể có \textbf{nhiều nhánh} khác nhau (mỗi nhánh ứng với một điều kiện).
Nếu chỉ dùng \texttt{if -- else} thì bạn thường phải lồng nhiều lớp \texttt{if}, khiến code dài và khó đọc.
Khi đó, cấu trúc \texttt{if -- else if -- else} giúp viết chương trình gọn hơn và rõ ràng hơn.

\subsection{Cú pháp và cách hoạt động}

\begin{lstlisting}
// The program checks conditions from top to bottom.
// It executes the first block whose condition is true, then stops checking.
if (condition_1) {
    // block 1
}
else if (condition_2) {
    // block 2
}
else if (condition_3) {
    // block 3
}
else {
    // default block (optional)
}
\end{lstlisting}

\textbf{Ghi nhớ nhanh:}
\begin{itemize}
    \item Có thể có \textbf{bao nhiêu} nhánh \texttt{else if} cũng được.
    \item Nhánh \texttt{else} \textbf{có thể có hoặc không}.
    \item Hệ thống kiểm tra điều kiện \textbf{từ trên xuống dưới}.
    \item Chỉ chạy \textbf{duy nhất 1 nhánh đầu tiên đúng}, sau đó \textbf{kết thúc} cấu trúc rẽ nhánh.
\end{itemize}

\vspace{2mm}

\subsection{Ví dụ 1: Xếp loại điểm bằng \texttt{else if}}

\begin{lstlisting}
#include <iostream>
using namespace std;

int main() {
    int score;
    cin >> score;

    // Classify the score into categories
    if (score >= 90) {
        cout << "Excellent\n";
    }
    else if (score >= 75) {
        cout << "Good\n";
    }
    else if (score >= 50) {
        cout << "Average\n";
    }
    else {
        cout << "Need improvement\n";
    }

    return 0;
}
\end{lstlisting}

\textbf{Input:}
\begin{verbatim}
76
\end{verbatim}
\textbf{Output:}
\begin{verbatim}
Good
\end{verbatim}

\vspace{3mm}

\subsection{Bài tập áp dụng}

\subsubsection{Bài 1: Số ngày của tháng}

\textbf{Yêu cầu:} Nhập tháng $m$ và năm $y$, in ra số ngày của tháng đó.
Chú ý: tháng 2 có 29 ngày nếu năm nhuận, 28 ngày nếu không nhuận.

\textbf{Quy tắc năm nhuận:}
\begin{itemize}
    \item Năm chia hết cho 400 $\Rightarrow$ năm nhuận, hoặc
    \item Năm chia hết cho 4 và không chia hết cho 100 $\Rightarrow$ năm nhuận.
\end{itemize}

\begin{lstlisting}
#include <iostream>
using namespace std;

int main() {
    int m, y;
    cin >> m >> y;

    // Months with 31 days
    if (m == 1 || m == 3 || m == 5 || m == 7 ||
        m == 8 || m == 10 || m == 12) {
        cout << 31 << "\n";
    }
    // Months with 30 days
    else if (m == 4 || m == 6 || m == 9 || m == 11) {
        cout << 30 << "\n";
    }
    // February
    else if (m == 2) {
        bool isLeap = (y % 400 == 0) || (y % 4 == 0 && y % 100 != 0);

        if (isLeap) cout << 29 << "\n";
        else cout << 28 << "\n";
    }
    // Invalid month
    else {
        cout << "Invalid month\n";
    }

    return 0;
}
\end{lstlisting}

\textbf{Ví dụ 1:}

\textbf{Input:}
\begin{verbatim}
2 2020
\end{verbatim}
\textbf{Output:}
\begin{verbatim}
29
\end{verbatim}

\textbf{Ví dụ 2:}

\textbf{Input:}
\begin{verbatim}
2 2021
\end{verbatim}
\textbf{Output:}
\begin{verbatim}
28
\end{verbatim}

\textbf{Ví dụ 3:}

\textbf{Input:}
\begin{verbatim}
13 2024
\end{verbatim}
\textbf{Output:}
\begin{verbatim}
Invalid month
\end{verbatim}

\vspace{3mm}

\subsubsection{Bài 2: Phân loại tam giác}

\textbf{Yêu cầu:} Nhập 3 cạnh nguyên $a,b,c$ (giả sử luôn tạo thành tam giác hợp lệ).
In:
\begin{itemize}
    \item 1 nếu tam giác đều
    \item 2 nếu tam giác cân
    \item 3 nếu tam giác vuông
    \item 4 nếu tam giác thường
\end{itemize}

\textbf{Ghi chú quan trọng:}
\begin{itemize}
    \item Kiểm tra ``đều'' trước vì ``đều'' cũng là một dạng của ``cân''.
    \item Kiểm tra ``vuông'' dùng định lý Pythagoras:
    $a^2=b^2+c^2$ hoặc $b^2=a^2+c^2$ hoặc $c^2=a^2+b^2$.
\end{itemize}

\begin{lstlisting}
#include <iostream>
using namespace std;

int main() {
    long long a, b, c;
    cin >> a >> b >> c;

    // Equilateral triangle
    if (a == b && b == c) {
        cout << 1 << "\n";
    }
    // Isosceles triangle
    else if (a == b || b == c || a == c) {
        cout << 2 << "\n";
    }
    // Right triangle (use long long to avoid overflow in squaring)
    else if (a * a == b * b + c * c ||
             b * b == a * a + c * c ||
             c * c == a * a + b * b) {
        cout << 3 << "\n";
    }
    // Scalene triangle
    else {
        cout << 4 << "\n";
    }

    return 0;
}
\end{lstlisting}

\textbf{Ví dụ 1:}

\textbf{Input:}
\begin{verbatim}
3 3 3
\end{verbatim}
\textbf{Output:}
\begin{verbatim}
1
\end{verbatim}

\textbf{Ví dụ 2:}

\textbf{Input:}
\begin{verbatim}
5 5 8
\end{verbatim}
\textbf{Output:}
\begin{verbatim}
2
\end{verbatim}

\textbf{Ví dụ 3:}

\textbf{Input:}
\begin{verbatim}
3 4 5
\end{verbatim}
\textbf{Output:}
\begin{verbatim}
3
\end{verbatim}

\textbf{Ví dụ 4:}

\textbf{Input:}
\begin{verbatim}
4 5 6
\end{verbatim}
\textbf{Output:}
\begin{verbatim}
4
\end{verbatim}

\section{Bảng mã ASCII và thư viện \texttt{cctype}}

\subsection{Bảng mã ASCII là gì?}

\textbf{ASCII} (American Standard Code for Information Interchange) là bảng mã gán \textbf{mỗi ký tự} (chữ cái, chữ số, ký hiệu, \dots) với một \textbf{số nguyên} tương ứng.

\begin{figure}[h]
    \centering
    \includegraphics[width=0.8\textwidth]{resource/img/ascii.png}
\end{figure}

Trong thực tế học C++ cơ bản, bạn chỉ cần nhớ một số cụm quan trọng để làm bài nhanh.

\subsubsection*{Các cụm ASCII nên nhớ}

\begin{center}
\renewcommand{\arraystretch}{1.2}
\begin{tabular}{|l|c|}
\hline
\textbf{Cụm ký tự} & \textbf{Mã ASCII (thập phân)} \\
\hline
\texttt{'a' -- 'z'} & 97 -- 122 \\
\texttt{'A' -- 'Z'} & 65 -- 90 \\
\texttt{'0' -- '9'} & 48 -- 57 \\
\hline
\end{tabular}
\end{center}

\textbf{Lưu ý quan trọng:} Trong C++, kiểu \texttt{char} lưu ký tự, nhưng bản chất nó vẫn là một số nhỏ (thường 1 byte).  
Khi bạn dùng \texttt{char} trong biểu thức số học, chương trình sẽ dùng \textbf{mã ASCII} của ký tự đó.

\vspace{2mm}

\subsection{Ví dụ: \texttt{char} hoạt động như số (ASCII)}

\subsubsection*{Ví dụ 1: In mã ASCII của ký tự}

\begin{lstlisting}
#include <bits/stdc++.h>
using namespace std;

int main() {
    char c1 = 'A', c2 = 'a', c3 = '0';

    // Print ASCII codes by casting to int
    cout << (int)c1 << " " << (int)c2 << " " << (int)c3 << "\n";

    int n = c1; // n becomes 65
    cout << n << "\n";

    // 'A' = 65, so 'A' + 100 = 165
    cout << (c1 + 100) << "\n";
    return 0;
}
\end{lstlisting}

\textbf{Output:}
\begin{verbatim}
65 97 48
65
165
\end{verbatim}

\subsubsection*{Ví dụ 2: Tăng/giảm ký tự}

\begin{lstlisting}
#include <bits/stdc++.h>
using namespace std;

int main() {
    char c = 'A';

    // 'A' = 65, so 65 + 10 = 75
    int n = c + 10;
    cout << n << "\n";

    // Pre-increment: 'A' -> 'B'
    ++c;
    cout << (int)c << "\n";
    cout << c << "\n";
    return 0;
}
\end{lstlisting}

\textbf{Output:}
\begin{verbatim}
75
66
B
\end{verbatim}

\vspace{3mm}

\subsection{Tự kiểm tra loại ký tự (không cần thư viện)}

Trong nhiều bài về chuỗi, bạn hay gặp các yêu cầu:
\textit{``ký tự này là chữ thường/chữ hoa/chữ số?''}
Bạn có thể tự kiểm tra bằng cách so sánh theo khoảng ASCII.

\subsubsection*{Kiểm tra chữ thường \texttt{'a'..'z'}}

\begin{lstlisting}
#include <bits/stdc++.h>
using namespace std;

int main() {
    char c = 'u';

    // Method 1: compare by characters
    if (c >= 'a' && c <= 'z') cout << "YES\n";
    else cout << "NO\n";

    // Method 2: compare by ASCII codes
    if (c >= 97 && c <= 122) cout << "YES\n";
    else cout << "NO\n";

    return 0;
}
\end{lstlisting}

\subsubsection*{Kiểm tra chữ hoa \texttt{'A'..'Z'}}

\begin{lstlisting}
#include <bits/stdc++.h>
using namespace std;

int main() {
    char c = 'u';

    if (c >= 'A' && c <= 'Z') cout << "YES\n";
    else cout << "NO\n";

    if (c >= 65 && c <= 90) cout << "YES\n";
    else cout << "NO\n";

    return 0;
}
\end{lstlisting}

\subsubsection*{Kiểm tra chữ cái (hoa hoặc thường)}

\begin{lstlisting}
#include <bits/stdc++.h>
using namespace std;

int main() {
    char c = 'Z';

    if ((c >= 'A' && c <= 'Z') || (c >= 'a' && c <= 'z')) cout << "YES\n";
    else cout << "NO\n";

    return 0;
}
\end{lstlisting}

\subsubsection*{Kiểm tra chữ số \texttt{'0'..'9'}}

\begin{lstlisting}
#include <bits/stdc++.h>
using namespace std;

int main() {
    char c = '5';

    // Method 1
    if (c >= '0' && c <= '9') cout << "YES\n";
    else cout << "NO\n";

    // Method 2
    if (c >= 48 && c <= 57) cout << "YES\n";
    else cout << "NO\n";

    return 0;
}
\end{lstlisting}

\vspace{2mm}

\subsubsection*{Đổi chữ hoa $\leftrightarrow$ chữ thường bằng ASCII}

Trong ASCII, chữ thường có mã \textbf{lớn hơn} chữ hoa tương ứng \textbf{32 đơn vị}:
\[
\texttt{'a'} - \texttt{'A'} = 97 - 65 = 32
\]
\begin{itemize}
    \item Hoa $\rightarrow$ thường: cộng 32
    \item Thường $\rightarrow$ hoa: trừ 32
\end{itemize}

\begin{lstlisting}
#include <bits/stdc++.h>
using namespace std;

int main() {
    char c = 'A';
    c += 32; // 'A' -> 'a'
    cout << c << "\n";

    char d = 'z';
    d -= 32; // 'z' -> 'Z'
    cout << d << "\n";

    return 0;
}
\end{lstlisting}

\textbf{Output:}
\begin{verbatim}
a
Z
\end{verbatim}

\vspace{3mm}

\subsection{Thư viện \texttt{cctype}}

C++ có sẵn các hàm để kiểm tra/chuyển đổi ký tự trong thư viện \texttt{<cctype>}
(\texttt{<ctype.h>} cũng dùng được).

\begin{center}
\renewcommand{\arraystretch}{1.2}
\begin{tabular}{|l|l|}
\hline
\textbf{Hàm} & \textbf{Ý nghĩa} \\
\hline
\texttt{islower(c)} & 1 nếu \texttt{c} là chữ thường, ngược lại 0 \\
\texttt{isupper(c)} & 1 nếu \texttt{c} là chữ hoa, ngược lại 0 \\
\texttt{isalpha(c)} & 1 nếu \texttt{c} là chữ cái, ngược lại 0 \\
\texttt{isdigit(c)} & 1 nếu \texttt{c} là chữ số, ngược lại 0 \\
\texttt{isalnum(c)} & 1 nếu \texttt{c} là chữ cái hoặc chữ số, ngược lại 0 \\
\texttt{tolower(c)} & trả về ký tự thường tương ứng (nếu có) \\
\texttt{toupper(c)} & trả về ký tự hoa tương ứng (nếu có) \\
\hline
\end{tabular}
\end{center}

\textbf{Lưu ý nhỏ khi dùng \texttt{cctype}:} Các hàm như \texttt{isupper}, \texttt{tolower} nhận \texttt{int} (thực chất là giá trị ký tự), nên trong C++ an toàn nhất là ép \texttt{unsigned char} trước khi gọi nếu dữ liệu có thể ngoài ASCII. Với newbie làm bài cơ bản ASCII thì cứ dùng trực tiếp cũng ổn.

\vspace{2mm}

\subsubsection*{Ví dụ 1: Kiểm tra chữ hoa/chữ thường}

\begin{lstlisting}
#include <bits/stdc++.h>
using namespace std;

int main() {
    char c = 'A', d = 'z';

    if (isupper(c)) cout << "YES\n";
    else cout << "NO\n";

    if (islower(d)) cout << "YES\n";
    else cout << "NO\n";

    return 0;
}
\end{lstlisting}

\textbf{Output:}
\begin{verbatim}
YES
YES
\end{verbatim}

\subsubsection*{Ví dụ 2: Chuyển đổi hoa/thường}

\begin{lstlisting}
#include <bits/stdc++.h>
using namespace std;

int main() {
    char c = 'A', d = 'z';

    c = tolower(c); // 'A' -> 'a'
    d = toupper(d); // 'z' -> 'Z'

    cout << c << " " << d << "\n";
    return 0;
}
\end{lstlisting}

\textbf{Output:}
\begin{verbatim}
a Z
\end{verbatim}

\section{Bài tập}

\begin{baitap}{Phân loại nấm}{https://marisaoj.com/problem/396}
Marisa đang phân tích một cây nấm có chỉ số độc tố \texttt{T} (từ \texttt{0.0} đến \texttt{10.0}).

\begin{itemize}
  \item Rất độc nếu \texttt{T} từ \texttt{9.0} trở lên.
  \item Độc nếu \texttt{T} từ \texttt{5.0} đến \texttt{8.9}.
  \item An toàn nếu \texttt{T} dưới \texttt{5.0}.
\end{itemize}

\noindent \textbf{Input}

Một dòng gồm một số thực \texttt{T} (có đúng 1 chữ số thập phân).

\noindent \textbf{Output}

\begin{itemize}
  \item In \texttt{VERY TOXIC} nếu nấm rất độc.
  \item In \texttt{TOXIC} nếu nấm độc.
  \item In \texttt{SAFE} nếu nấm an toàn.
\end{itemize}

\noindent \textbf{Ví dụ}

\begin{simple_example}
5.0 & TOXIC \\
\end{simple_example}

\begin{huonggiai}
  \item Đọc số thực \texttt{T}.
  \item So sánh theo ngưỡng:
  \begin{itemize}
    \item Nếu \texttt{T >= 9.0} $\rightarrow$ \texttt{VERY TOXIC}.
    \item Ngược lại nếu \texttt{T >= 5.0} $\rightarrow$ \texttt{TOXIC}.
    \item Còn lại $\rightarrow$ \texttt{SAFE}.
  \end{itemize}
  \item Vì dữ liệu có đúng 1 chữ số thập phân nên so sánh trực tiếp kiểu \texttt{double} là đủ an toàn.
\end{huonggiai}

\noindent \textbf{Cài đặt}
\begin{lstlisting}
#include <bits/stdc++.h>
#define int long long
using namespace std;

signed main(){
    double T;
    cin >> T;

    if(T >= 9.0){
        cout << "VERY TOXIC";
    }else if(T >= 5.0){
        cout << "TOXIC";
    }else{
        cout << "SAFE";
    }

    return 0;
}
\end{lstlisting}

\end{baitap}

\begin{baitap}{Phương trình nghiệm nguyên}{https://marisaoj.com/problem/397}
Tìm nghiệm nguyên của phương trình:
\[
a x + b = 0.
\]
Trong đó \textbf{nghiệm nguyên} nghĩa là \(x\) phải là một số nguyên.

\noindent \textbf{Input}

Một dòng gồm hai số nguyên \texttt{a, b}.

\noindent \textbf{Output}

\begin{itemize}
  \item Nếu có đúng một nghiệm nguyên, in ra nghiệm đó.
  \item Nếu có vô số nghiệm, in ra \texttt{INFINITE SOLUTIONS}.
  \item Nếu không tồn tại nghiệm nguyên, in ra \texttt{NO SOLUTION}.
\end{itemize}

\noindent \textbf{Điều kiện}

\[
-100 \le a, b \le 100
\]

\noindent \textbf{Ví dụ}

\begin{simple_example}
3 6 & -2 \\
\end{simple_example}

\begin{huonggiai}
  \item Xét các trường hợp của hệ số \texttt{a}:
  \begin{itemize}
    \item Nếu \(a = 0\):
      \begin{itemize}
        \item Nếu \(b = 0\) thì phương trình \(0x + 0 = 0\) đúng với mọi \(x\) $\Rightarrow$ vô số nghiệm.
        \item Nếu \(b \ne 0\) thì \(0x + b = 0\) vô lý $\Rightarrow$ không có nghiệm.
      \end{itemize}
    \item Nếu \(a \ne 0\):
      \begin{itemize}
        \item Ta có \(x = -\dfrac{b}{a}\).
        \item Để \(x\) là số nguyên thì \(b\) phải chia hết cho \(a\), tức là \(b \bmod a = 0\).
      \end{itemize}
  \end{itemize}
\end{huonggiai}

\noindent \textbf{Cài đặt}
\begin{lstlisting}
#include <bits/stdc++.h>
#define int long long
using namespace std;

signed main(){
    int a, b;
    cin >> a >> b;

    if(a == 0){
        if(b == 0) cout << "INFINITE SOLUTIONS";
        else cout << "NO SOLUTION";
    }else{
        if(b % a == 0){
            cout << (-b / a);
        }else{
            cout << "NO SOLUTION";
        }
    }

    return 0;
}
\end{lstlisting}

\end{baitap}

\begin{baitap}{Tuổi}{https://marisaoj.com/problem/420}
Có hai người:
\begin{itemize}
  \item Người thứ nhất sinh ngày \texttt{a}, tháng \texttt{b}, năm \texttt{c}.
  \item Người thứ hai sinh ngày \texttt{x}, tháng \texttt{y}, năm \texttt{z}.
\end{itemize}
Biết rằng hai người không sinh cùng ngày. Hãy xác định người nào nhiều tuổi hơn.

\noindent \textbf{Input}

Một dòng gồm sáu số nguyên \texttt{a, b, c, x, y, z}.

\noindent \textbf{Output}

In ra \texttt{1} nếu người thứ nhất nhiều tuổi hơn, ngược lại in ra \texttt{2}.

\noindent \textbf{Điều kiện}

Đảm bảo ngày hợp lệ và hai người không sinh cùng ngày.

\noindent \textbf{Ví dụ}

\begin{simple_example}
7 10 1990 8 10 1990 & 1 \\
\end{simple_example}

\begin{huonggiai}
  \item Người nhiều tuổi hơn là người có ngày sinh \textbf{sớm hơn}.
  \item So sánh theo thứ tự:
  \begin{itemize}
    \item So sánh năm: năm nhỏ hơn $\Rightarrow$ sinh sớm hơn.
    \item Nếu năm bằng nhau, so sánh tháng: tháng nhỏ hơn $\Rightarrow$ sinh sớm hơn.
    \item Nếu tháng bằng nhau, so sánh ngày: ngày nhỏ hơn $\Rightarrow$ sinh sớm hơn.
  \end{itemize}
  \item Nếu ngày sinh của người thứ nhất sớm hơn người thứ hai $\Rightarrow$ in \texttt{1}, ngược lại in \texttt{2}.
\end{huonggiai}

\noindent \textbf{Cài đặt}
\begin{lstlisting}
#include <bits/stdc++.h>
#define int long long
using namespace std;

signed main(){
    int a, b, c, x, y, z;
    cin >> a >> b >> c >> x >> y >> z;

    // so sanh nam truoc
    if(c != z){
        if(c < z) cout << 1;
        else cout << 2;
        return 0;
    }

    // neu nam bang nhau, so sanh thang
    if(b != y){
        if(b < y) cout << 1;
        else cout << 2;
        return 0;
    }

    // neu thang bang nhau, so sanh ngay
    if(a < x) cout << 1;
    else cout << 2;

    return 0;
}
\end{lstlisting}

\end{baitap}

\begin{baitap}{Ghép hình chữ nhật}{https://marisaoj.com/problem/421}
Cho hai hình chữ nhật có kích thước lần lượt \texttt{a×b} và \texttt{c×d}.
Kiểm tra xem có thể ghép hai hình chữ nhật này thành một hình chữ nhật khác bằng cách đặt chúng cạnh nhau hay không.

\noindent \textbf{Input}

Một dòng gồm bốn số nguyên \texttt{a, b, c, d}.

\noindent \textbf{Output}

In ra \texttt{YES} nếu có thể ghép, ngược lại in ra \texttt{NO}.

\noindent \textbf{Điều kiện}

\[
1 \le a, b, c, d \le 1000
\]

\noindent \textbf{Ví dụ}

\begin{simple_example}
1 2 3 4 & NO \\
\end{simple_example}

\begin{huonggiai}
  \item Ghép ``cạnh nhau'' có 2 kiểu:
  \begin{itemize}
    \item Ghép ngang: hai hình phải có \textbf{cùng chiều cao}.
    \item Ghép dọc: hai hình phải có \textbf{cùng chiều rộng}.
  \end{itemize}
  \item Mỗi hình có thể xoay, tức là \((a,b)\) hoặc \((b,a)\), và \((c,d)\) hoặc \((d,c)\).
  \item Ta xét 4 cách xoay, với mỗi cách kiểm tra:
  \begin{itemize}
    \item \(H_1 = H_2\) (ghép ngang) hoặc \(W_1 = W_2\) (ghép dọc).
  \end{itemize}
\end{huonggiai}

\noindent \textbf{Cài đặt}
\begin{lstlisting}
#include <bits/stdc++.h>
#define int long long
using namespace std;

signed main(){
    int a, b, c, d;
    cin >> a >> b >> c >> d;

    int ok = 0;

    // Case 1: (a,b) va (c,d)
    if(b == d) ok = 1; // ghep ngang (cao bang nhau)
    if(a == c) ok = 1; // ghep doc (rong bang nhau)

    // Case 2: (a,b) va (d,c)
    if(b == c) ok = 1;
    if(a == d) ok = 1;

    // Case 3: (b,a) va (c,d)
    if(a == d) ok = 1;
    if(b == c) ok = 1;

    // Case 4: (b,a) va (d,c)
    if(a == c) ok = 1;
    if(b == d) ok = 1;

    cout << (ok ? "YES" : "NO");
    return 0;
}
\end{lstlisting}

\end{baitap}


\begin{baitap}{Tam giác}{https://marisaoj.com/problem/4}
Nhập vào ba số nguyên \texttt{a, b, c}. Hãy kiểm tra xem có thể tạo được một tam giác từ ba cạnh có độ dài \texttt{a, b, c} hay không.

\noindent \textbf{Input}

Một dòng gồm 3 số nguyên \texttt{a, b, c}.

\noindent \textbf{Output}

In ra \texttt{YES} nếu có thể tạo thành tam giác, ngược lại in ra \texttt{NO}.

\noindent \textbf{Điều kiện}

\[
1 \le a, b, c \le 1000
\]

\noindent \textbf{Ví dụ}

\begin{simple_example}
3 4 5 & YES \\
\end{simple_example}

\begin{huonggiai}
  \item Ba đoạn thẳng tạo thành tam giác khi và chỉ khi \textbf{tổng độ dài hai cạnh bất kỳ lớn hơn cạnh còn lại}.
  \item Tức là cần đồng thời thỏa:
  \[
  a+b>c,\quad a+c>b,\quad b+c>a.
  \]
  \item Nếu cả 3 bất đẳng thức đều đúng thì in \texttt{YES}, ngược lại in \texttt{NO}.
\end{huonggiai}

\noindent \textbf{Cài đặt}
\begin{lstlisting}
#include <bits/stdc++.h>
#define int long long
using namespace std;

signed main(){
    int a, b, c;
    cin >> a >> b >> c;

    if(a + b > c && a + c > b && b + c > a){
        cout << "YES";
    }else{
        cout << "NO";
    }

    return 0;
}
\end{lstlisting}

\end{baitap}

\begin{baitap}{Máy tính}{https://marisaoj.com/problem/535}
Thiết kế một chiếc máy tính xử lí được bốn phép toán cơ bản: cộng (\texttt{+}), trừ (\texttt{-}),
nhân (\texttt{*}) và chia (\texttt{/}) với hai số thực \texttt{a}, \texttt{b}.

\noindent \textbf{Input}

Một dòng gồm lần lượt:
\begin{itemize}
  \item Số thực \texttt{a}
  \item Một kí tự biểu diễn phép toán
  \item Số thực \texttt{b}
\end{itemize}

\noindent \textbf{Output}

\begin{itemize}
  \item In ra kết quả của phép tính, làm tròn tới \texttt{3} chữ số thập phân.
  \item Nếu phép tính không thể thực hiện được (chia cho 0), in ra \texttt{ze}.
\end{itemize}

\noindent \textbf{Điều kiện}

\[
-1000 \le a, b \le 1000
\]

\noindent \textbf{Ví dụ}

\begin{simple_example}
4 * 5 & 20.000 \\
\end{simple_example}

\begin{huonggiai}
  \item Đọc hai số thực \texttt{a}, \texttt{b} và kí tự phép toán.
  \item Xét từng trường hợp của phép toán:
  \begin{itemize}
    \item \texttt{+}: tính \(a + b\)
    \item \texttt{-}: tính \(a - b\)
    \item \texttt{*}: tính \(a \times b\)
    \item \texttt{/}: nếu \(b = 0\) thì không thực hiện được
  \end{itemize}
  \item Nếu phép toán hợp lệ, in kết quả với định dạng \texttt{fixed} và \texttt{setprecision(3)}.
\end{huonggiai}

\noindent \textbf{Cài đặt}
\begin{lstlisting}
#include <bits/stdc++.h>
#define int long long
using namespace std;

signed main(){
    double a, b;
    char op;
    cin >> a >> op >> b;

    double res;
    bool ok = true;

    if(op == '+'){
        res = a + b;
    }else if(op == '-'){
        res = a - b;
    }else if(op == '*'){
        res = a * b;
    }else if(op == '/'){
        if(b == 0){
            ok = false;
        }else{
            res = a / b;
        }
    }else{
        ok = false;
    }

    if(!ok){
        cout << "ze";
    }else{
        cout << fixed << setprecision(3) << res;
    }

    return 0;
}
\end{lstlisting}

\end{baitap}

\begin{baitap}{Nhỏ nhất và lớn nhất}{https://marisaoj.com/problem/5}
Cho ba số nguyên \texttt{a, b, c}. Hãy in ra số nhỏ nhất và số lớn nhất trong ba số đó.

\noindent \textbf{Input}

Một dòng gồm 3 số nguyên \texttt{a, b, c}.

\noindent \textbf{Output}

In ra hai số: số nhỏ nhất và số lớn nhất.

\noindent \textbf{Giới hạn}

\[
1 \le a, b, c \le 1000
\]

\noindent \textbf{Ví dụ}

\begin{simple_example}
4 3 4 & 3 4 \\
\end{simple_example}

\begin{huonggiai}
  \item Khởi tạo \texttt{mn} và \texttt{mx} lần lượt bằng \texttt{a}.
  \item So sánh \texttt{b} và \texttt{c} với \texttt{mn} để cập nhật giá trị nhỏ nhất.
  \item So sánh \texttt{b} và \texttt{c} với \texttt{mx} để cập nhật giá trị lớn nhất.
\end{huonggiai}

\noindent \textbf{Cài đặt}
\begin{lstlisting}
#include <bits/stdc++.h>
#define int long long
using namespace std;

signed main(){
    int a, b, c;
    cin >> a >> b >> c;

    int mn = a, mx = a;

    if(b < mn) mn = b;
    if(c < mn) mn = c;

    if(b > mx) mx = b;
    if(c > mx) mx = c;

    cout << mn << " " << mx;
    return 0;
}
\end{lstlisting}

\end{baitap}

\begin{baitap}{Hình chữ nhật}{https://marisaoj.com/problem/7}
Cho độ dài ba cạnh \texttt{a, b, c} của một hình chữ nhật.
Biết rằng trong ba cạnh này có hai cạnh bằng nhau (là hai cạnh đối diện của hình chữ nhật).
Hãy in ra độ dài của cạnh còn lại.

\noindent \textbf{Input}

Một dòng gồm ba số nguyên \texttt{a, b, c}.

\noindent \textbf{Output}

In ra độ dài cạnh còn lại của hình chữ nhật.

\noindent \textbf{Điều kiện}

\[
1 \le a, b, c \le 1000
\]

Không có trường hợp không hợp lệ.

\noindent \textbf{Ví dụ}

\begin{simple_example}
4 3 4 & 3 \\
\end{simple_example}

\begin{huonggiai}
  \item Trong hình chữ nhật, hai cạnh đối diện có độ dài bằng nhau.
  \item Do đó trong ba số \texttt{a, b, c}:
  \begin{itemize}
    \item Nếu \texttt{a = b} thì cạnh còn lại là \texttt{c}.
    \item Nếu \texttt{a = c} thì cạnh còn lại là \texttt{b}.
    \item Ngược lại, \texttt{b = c} và cạnh còn lại là \texttt{a}.
  \end{itemize}
\end{huonggiai}

\noindent \textbf{Cài đặt}
\begin{lstlisting}
#include <bits/stdc++.h>
#define int long long
using namespace std;

signed main(){
    int a, b, c;
    cin >> a >> b >> c;

    if(a == b){
        cout << c;
    }else if(a == c){
        cout << b;
    }else{
        cout << a;
    }

    return 0;
}
\end{lstlisting}

\end{baitap}

\begin{baitap}{Đoạn thẳng}{https://marisaoj.com/problem/419}
Cho hai đoạn thẳng \texttt{AB} và \texttt{CD} nằm trên trục số.
Đoạn \texttt{AB} kéo dài từ \texttt{a} đến \texttt{b}, đoạn \texttt{CD} kéo dài từ \texttt{c} đến \texttt{d}.
Hãy kiểm tra xem hai đoạn thẳng có điểm chung không.

\noindent \textbf{Input}

Một dòng gồm bốn số nguyên \texttt{a, b, c, d}.

\noindent \textbf{Output}

Nếu hai đoạn thẳng có điểm chung, in ra \texttt{YES}. Ngược lại in ra \texttt{NO}.

\noindent \textbf{Điều kiện}

\[
1 \le a \le b \le 1000,\quad 1 \le c \le d \le 1000
\]

\noindent \textbf{Ví dụ}

\begin{simple_example}
1 2 2 3 & YES \\
\end{simple_example}

\begin{huonggiai}
  \item Hai đoạn thẳng \([a,b]\) và \([c,d]\) có điểm chung khi chúng \textbf{giao nhau}.
  \item Chúng \textbf{không} có điểm chung chỉ khi một đoạn nằm hoàn toàn bên trái đoạn kia:
  \[
  b < c \quad \text{hoặc} \quad d < a.
  \]
  \item Nếu không rơi vào hai trường hợp trên thì chắc chắn có ít nhất một điểm chung $\Rightarrow$ in \texttt{YES}.
\end{huonggiai}

\noindent \textbf{Cài đặt}
\begin{lstlisting}
#include <bits/stdc++.h>
#define int long long
using namespace std;

signed main(){
    int a, b, c, d;
    cin >> a >> b >> c >> d;

    if(b < c || d < a){
        cout << "NO";
    }else{
        cout << "YES";
    }

    return 0;
}
\end{lstlisting}

\end{baitap}

\begin{baitap}{Tăng dần}{https://marisaoj.com/problem/6}
Cho ba số nguyên \texttt{a, b, c}. Hãy in ra ba số theo thứ tự từ bé đến lớn.

\noindent \textbf{Input}

Một dòng gồm 3 số nguyên \texttt{a, b, c}.

\noindent \textbf{Output}

In ra 3 số theo thứ tự tăng dần.

\noindent \textbf{Giới hạn}

\[
1 \le a, b, c \le 1000
\]

\noindent \textbf{Ví dụ}

\begin{simple_example}
8 3 9 & 3 8 9 \\
\end{simple_example}

\begin{huonggiai}
  \item Ta sắp xếp ba số bằng cách so sánh và đổi chỗ.
  \item Nếu \texttt{a > b} thì đổi chỗ \texttt{a} và \texttt{b}.
  \item Nếu \texttt{a > c} thì đổi chỗ \texttt{a} và \texttt{c}.
  \item Nếu \texttt{b > c} thì đổi chỗ \texttt{b} và \texttt{c}.
  \item Sau các bước trên, ta có \(\texttt{a} \le \texttt{b} \le \texttt{c}\).
\end{huonggiai}

\noindent \textbf{Cài đặt}
\begin{lstlisting}
#include <bits/stdc++.h>
#define int long long
using namespace std;

signed main(){
    int a, b, c;
    cin >> a >> b >> c;

    if(a > b){
        int t = a; a = b; b = t;
    }
    if(a > c){
        int t = a; a = c; c = t;
    }
    if(b > c){
        int t = b; b = c; c = t;
    }

    cout << a << " " << b << " " << c;
    return 0;
}
\end{lstlisting}

\end{baitap}

\begin{baitap}{Số chính phương}{https://marisaoj.com/problem/11}
Cho số nguyên \texttt{a}. Hãy xác định xem \texttt{a} có phải là số chính phương hay không.

\noindent \textbf{Input}

Một số nguyên \texttt{a}.

\noindent \textbf{Output}

In ra \texttt{YES} nếu \texttt{a} là số chính phương, ngược lại in ra \texttt{NO}.

\noindent \textbf{Điều kiện}

\[
1 \le a \le 10^{18}
\]

\noindent \textbf{Ví dụ}

\begin{simple_example}
16 & YES \\
\end{simple_example}

\begin{huonggiai}
  \item Một số là số chính phương nếu tồn tại số nguyên \texttt{x} sao cho \(\texttt{x}^2 = a\).
  \item Ta lấy căn bậc hai của \texttt{a}, làm tròn xuống để được \texttt{x}.
  \item Nếu \(\texttt{x} \times \texttt{x} = a\) thì \texttt{a} là số chính phương.
  \item Cách này chạy rất nhanh và phù hợp với \texttt{a} lớn tới \(10^{18}\).
\end{huonggiai}

\noindent \textbf{Cài đặt}
\begin{lstlisting}
#include <bits/stdc++.h>
#define int long long
using namespace std;

signed main(){
    int a;
    cin >> a;

    int x = sqrt((long double)a);
    if(x * x == a){
        cout << "YES";
    }else{
        cout << "NO";
    }

    return 0;
}
\end{lstlisting}

\end{baitap}

\begin{baitap}{Chữ hoa chữ thường}{https://marisaoj.com/problem/13}
Cho một chữ cái \texttt{c}. Nếu \texttt{c} là chữ thường, hãy in ra \texttt{c} dưới dạng chữ hoa, và ngược lại.

\noindent \textbf{Input}

Một chữ cái \texttt{c} (chữ cái trong bảng chữ cái tiếng Anh).

\noindent \textbf{Output}

Nếu \texttt{c} là chữ hoa, in ra \texttt{c} dưới dạng chữ thường;  
nếu \texttt{c} là chữ thường, in ra \texttt{c} dưới dạng chữ hoa.

\noindent \textbf{Ví dụ}

\begin{simple_example}
C & c \\
\end{simple_example}

\begin{huonggiai}
  \item Trong bảng mã ASCII:
  \begin{itemize}
    \item Chữ hoa nằm trong đoạn từ \texttt{'A'} đến \texttt{'Z'}.
    \item Chữ thường nằm trong đoạn từ \texttt{'a'} đến \texttt{'z'}.
  \end{itemize}
  \item Nếu \texttt{c} là chữ hoa, ta đổi sang chữ thường bằng cách cộng hiệu \texttt{'a' - 'A'}.
  \item Nếu \texttt{c} là chữ thường, ta đổi sang chữ hoa bằng cách trừ hiệu \texttt{'a' - 'A'}.
\end{huonggiai}

\noindent \textbf{Cài đặt}
\begin{lstlisting}
#include <bits/stdc++.h>
#define int long long
using namespace std;

signed main(){
    char c;
    cin >> c;

    if(c >= 'A' && c <= 'Z'){
        // doi sang chu thuong
        c = c + ('a' - 'A');
    }else{
        // doi sang chu hoa
        c = c - ('a' - 'A');
    }

    cout << c;
    return 0;
}
\end{lstlisting}

\end{baitap}

\begin{baitap}{Đếm chữ}{https://marisaoj.com/problem/14}
Cho hai chữ cái \texttt{a}, \texttt{b} (có thể là chữ hoa hoặc chữ thường).
Hãy đếm số lượng chữ cái \textbf{nằm giữa} \texttt{a} và \texttt{b} trong bảng chữ cái tiếng Anh.

\noindent \textbf{Input}

Một dòng gồm 2 chữ cái \texttt{a, b}.

\noindent \textbf{Output}

In ra số lượng chữ cái giữa \texttt{a} và \texttt{b} trong bảng chữ cái tiếng Anh.

\noindent \textbf{Ví dụ}

\begin{simple_example}
a E & 3 \\
\end{simple_example}

\begin{huonggiai}
  \item Chuyển cả hai chữ cái về cùng một dạng (ví dụ: chữ thường) để dễ so sánh.
  \item Khi đã cùng dạng, vị trí của chữ cái trong bảng chữ cái có thể tính bằng:
  \[
  pos = c - 'a'
  \]
  (với \texttt{a} có vị trí 0, \texttt{b} có vị trí 1, \dots).
  \item Số chữ cái nằm giữa hai chữ là:
  \[
  |pos(a) - pos(b)| - 1.
  \]
  \item Ví dụ: \texttt{a} và \texttt{e} có khoảng cách 4, nên ở giữa có \(4-1=3\) chữ: \texttt{b, c, d}.
\end{huonggiai}

\noindent \textbf{Cài đặt}
\begin{lstlisting}
#include <bits/stdc++.h>
#define int long long
using namespace std;

signed main(){
    char a, b;
    cin >> a >> b;

    // chuyen ve chu thuong neu la chu hoa
    if(a >= 'A' && a <= 'Z') a = a + ('a' - 'A');
    if(b >= 'A' && b <= 'Z') b = b + ('a' - 'A');

    int posa = a - 'a';
    int posb = b - 'a';

    int ans = llabs(posa - posb) - 1;
    cout << ans;

    return 0;
}
\end{lstlisting}

\end{baitap}

\begin{baitap}{Định dạng thời gian}{https://marisaoj.com/problem/416}
Hãy đổi từ \texttt{d} giây sang dạng giờ, phút, giây.

\noindent \textbf{Input}

Một dòng gồm số nguyên \texttt{d} (số giây).

\noindent \textbf{Output}

In ra ba số nguyên \texttt{h, m, s} lần lượt là số giờ, phút và giây tương ứng.

\noindent \textbf{Điều kiện}

\[
1 \le d \le 10^{18}
\]

\noindent \textbf{Ví dụ}

\begin{simple_example}
3929 & 1 5 29 \\
\end{simple_example}

\begin{huonggiai}
  \item Một giờ có \(3600\) giây, một phút có \(60\) giây.
  \item Số giờ:
  \[
  h = \left\lfloor \dfrac{d}{3600} \right\rfloor
  \]
  \item Số giây còn lại sau khi lấy giờ:
  \[
  d = d \bmod 3600
  \]
  \item Số phút:
  \[
  m = \left\lfloor \dfrac{d}{60} \right\rfloor
  \]
  \item Số giây còn lại chính là:
  \[
  s = d \bmod 60
  \]
\end{huonggiai}

\noindent \textbf{Cài đặt}
\begin{lstlisting}
#include <bits/stdc++.h>
#define int long long
using namespace std;

signed main(){
    int d;
    cin >> d;

    int h = d / 3600;
    d %= 3600;

    int m = d / 60;
    int s = d % 60;

    cout << h << " " << m << " " << s;
    return 0;
}
\end{lstlisting}

\end{baitap}

\begin{baitap}{Hóa đơn tiền điện}{https://marisaoj.com/problem/587}
Tiền điện hàng tháng được tính theo các mức sau:
\begin{itemize}
  \item Từ kWh \texttt{0} đến \texttt{50}: mỗi kWh giá \texttt{a} đồng.
  \item Từ kWh \texttt{51} đến \texttt{100}: mỗi kWh giá \texttt{b} đồng.
  \item Từ kWh \texttt{101} đến \texttt{150}: mỗi kWh giá \texttt{c} đồng.
  \item Từ kWh \texttt{151} trở đi: mỗi kWh giá \texttt{d} đồng.
\end{itemize}

Cho số kWh sử dụng là \texttt{x} và bốn số nguyên \texttt{a, b, c, d}.  
Hãy tính tổng số tiền điện phải trả.

\noindent \textbf{Input}

Một dòng gồm năm số nguyên \texttt{x, a, b, c, d}.

\noindent \textbf{Output}

In ra một số nguyên là số tiền điện.

\noindent \textbf{Điều kiện}

\[
1 \le x, a, b, c, d \le 1000
\]

\noindent \textbf{Ví dụ}

\begin{simple_example}
60 1 2 3 4 & 70 \\
\end{simple_example}

\begin{huonggiai}
  \item Tiền điện được tính theo từng bậc, không phải tất cả kWh đều tính cùng một giá.
  \item Ta lần lượt trừ số kWh đã tính ở các mức trước:
  \begin{itemize}
    \item Nếu \texttt{x} lớn hơn 50, tính 50 kWh đầu với giá \texttt{a}.
    \item Nếu còn vượt 100, tính tiếp 50 kWh với giá \texttt{b}.
    \item Nếu còn vượt 150, tính tiếp 50 kWh với giá \texttt{c}.
    \item Phần còn lại (nếu có) tính với giá \texttt{d}.
  \end{itemize}
\end{huonggiai}

\noindent \textbf{Cài đặt}
\begin{lstlisting}
#include <bits/stdc++.h>
#define int long long
using namespace std;

signed main(){
    int x, a, b, c, d;
    cin >> x >> a >> b >> c >> d;

    int cost = 0;

    if(x <= 50){
        cost = x * a;
    }else if(x <= 100){
        cost = 50 * a + (x - 50) * b;
    }else if(x <= 150){
        cost = 50 * a + 50 * b + (x - 100) * c;
    }else{
        cost = 50 * a + 50 * b + 50 * c + (x - 150) * d;
    }

    cout << cost;
    return 0;
}
\end{lstlisting}

\end{baitap}

\begin{baitap}{Ăn nấm}{https://marisaoj.com/problem/588}
Vào những ngày \textbf{không phải cuối tuần} (từ thứ Hai đến thứ Sáu), Marisa sẽ ăn một cây nấm.
Marisa bắt đầu ăn nấm từ \texttt{thứ x}, hỏi sau \texttt{y} ngày Marisa đã ăn bao nhiêu nấm.

\noindent \textbf{Input}

Một dòng gồm hai số nguyên \texttt{x, y}.

\noindent \textbf{Output}

In ra số nấm Marisa đã ăn.

\noindent \textbf{Điều kiện}

\[
2 \le x \le 8 \ (8 \text{ là Chủ Nhật}), \qquad 1 \le y \le 10^6
\]

\noindent \textbf{Ví dụ}

\begin{simple_example}
6 5 & 3 \\
\end{simple_example}

\begin{huonggiai}
  \item Quy ước: \texttt{2..8} tương ứng Thứ Hai..Chủ Nhật.
  \item Gọi \(\texttt{full} = \lfloor y/7 \rfloor\) là số tuần đầy đủ, khi đó ăn được \(\texttt{full} \times 5\) nấm.
  \item Phần còn lại \(\texttt{rem} = y \bmod 7\) chỉ nằm trong 7 ngày đầu của một tuần, có thể tính bằng công thức:
  \begin{itemize}
    \item Đặt \(s = x - 2\) (đổi về chỉ số 0..6), với 0..4 là ngày ăn, 5..6 là cuối tuần.
    \item Trong \texttt{rem} ngày, ta xét các chỉ số ngày \(s, s+1, \dots, s+\texttt{rem}-1\) theo modulo 7.
    \item Số ngày ăn bằng \(\texttt{rem} -\) (số ngày rơi vào cuối tuần).
  \end{itemize}
  \item Số ngày cuối tuần trong đoạn dài \texttt{rem} được tính bằng cách đếm số phần tử rơi vào tập \(\{5,6\}\).
\end{huonggiai}

\noindent \textbf{Cài đặt}
\begin{lstlisting}
#include <bits/stdc++.h>
#define int long long
using namespace std;

signed main(){
    int x, y;
    cin >> x >> y;

    int full = y / 7;
    int rem  = y % 7;

    int ans = full * 5;

    // s: 0..6 (0=Mon,1=Tue,...,4=Fri,5=Sat,6=Sun)
    int s = x - 2;

    // dem so ngay cuoi tuan trong doan rem ngay
    int weekend = 0;
    if(rem > 0){
        // neu doan rem khong vuot qua het tuan (s+rem-1 <= 6)
        if(s + rem - 1 <= 6){
            // giao voi [5,6]
            int L = s;
            int R = s + rem - 1;
            int l2 = max(L, 5LL);
            int r2 = min(R, 6LL);
            if(l2 <= r2) weekend = r2 - l2 + 1;
        }else{
            // doan bi "vong" qua tuan moi: tach thanh 2 doan
            // Doan 1: [s, 6], Doan 2: [0, (s+rem-1)%7]
            int L1 = s, R1 = 6;
            int l2 = max(L1, 5LL);
            int r2 = min(R1, 6LL);
            if(l2 <= r2) weekend += r2 - l2 + 1;

            int end2 = (s + rem - 1) % 7;
            int L2 = 0, R2 = end2;
            l2 = max(L2, 5LL);
            r2 = min(R2, 6LL);
            if(l2 <= r2) weekend += r2 - l2 + 1;
        }
    }

    ans += (rem - weekend);
    cout << ans;
    return 0;
}
\end{lstlisting}

\end{baitap}
